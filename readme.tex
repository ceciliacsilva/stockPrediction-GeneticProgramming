% Created 2017-11-30 Qui 19:41
\documentclass[11pt]{article}
\usepackage[utf8]{inputenc}
\usepackage[T1]{fontenc}
\usepackage{fixltx2e}
\usepackage{graphicx}
\usepackage{longtable}
\usepackage{float}
\usepackage{wrapfig}
\usepackage{rotating}
\usepackage[normalem]{ulem}
\usepackage{amsmath}
\usepackage{textcomp}
\usepackage{marvosym}
\usepackage{wasysym}
\usepackage{amssymb}
\usepackage{hyperref}
\tolerance=1000
\author{Cecília Carneiro e Silva}
\date{23/11/2017}
\title{Genetic Programming - Stock Prediction}
\hypersetup{
  pdfkeywords={},
  pdfsubject={},
  pdfcreator={Emacs 25.1.1 (Org mode 8.2.10)}}
\begin{document}

\maketitle
\tableofcontents


\section{Use}
\label{sec-1}

\subsection{Run.rkt}
\label{sec-1-1}

Treinar o modelo:

\begin{verbatim}
> (train-stockPrediction "aapl" 50 "Apple50-b" #:np 110 #:endSimul 110)
\end{verbatim}

Um diretório com o nome do projeto será criado na pasta "output".

Executar um modelo já criado:

\begin{verbatim}
> (run-stockPrediction "Apple50-b")
\end{verbatim}

Treina o modelo com os valores mais novos que a data do último treinamento.

\subsection{htmlCreate.rkt}
\label{sec-1-2}

Para gerar o Relatório HTML:

\begin{verbatim}
> (html-responseStock "Apple50-b")
\end{verbatim}
% Emacs 25.1.1 (Org mode 8.2.10)
\end{document}
